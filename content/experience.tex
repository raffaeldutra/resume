% Liber Capital
\def\experienceLiberCapital{
  {%
    \IfLanguageName{portuguese}%
    {%
      Nuvem pública: Amazon Web Services (AWS), Google Cloud Platform (GCP).
    }
    {%
      Public Cloud: Amazon Web Services (AWS), Google Cloud Platform (GCP).
    }
  },
  {%
    \IfLanguageName{portuguese}%
    {%
      Infraestrutura como código utilizando Terraform com reuso de código utilizando módulos e gerência de configuração com Ansible.
    }
    {%
      Infrastructure as code using Terraform with code reuse using modules and configuration management using Ansible.
    }
  },
  {%
    \IfLanguageName{portuguese}%
    {%
      Monitoramento e Logging: Prometheus, Grafana, Fluentd e Datadog para diferentes cenários.
      \newline
      Métricas de sistemas, como latência, erros 5xx e 4xx, uso de memória, quantidades de requests e etc.
    }
    {%
      Monitoring and Logging: Prometheus, Grafana, Fluentd and Datadog for different scenarios.
      \newline
      Systems metrics, such as latency, 5xx and 4xx errors, memory usage, requests amoun and etc.
    }
  },
  {%
    \IfLanguageName{portuguese}%
    {%
      Serverless com Lambda e gerenciamento de APIs com API Gateway.
    }
    {%
      Serverless functions with Lambda and API management with API Gateway.
    }
  },
  {%
    \IfLanguageName{portuguese}%
    {%
      Desenvolvimento e manutenção de clusters Kubernetes para múltiplos ambientes, garantindo alta disponibilidade e escalabilidade de workloads usando KEDA
    }
    {%
      Development and maintenance of Kubernetes clusters for multiple environments, ensuring high availability and workload scalability using KEDA.
    }
  }
}

% 4ALL
\def\experienceFourAll{
  {%
    \IfLanguageName{portuguese}%
    {%
      Infraestrutura como código.
      \newline
      Diversos projetos criados com Terrafom para ambientes AWS.
    }
    {%
      Infrastructure as code.
      \newline
      Many projects created using Terraform for AWS environments.
    }
  },
  {%
    \IfLanguageName{portuguese}%
    {%
      Implementação e administração de clusters Kubernetes (EKS).
      \newline
      Clusters em diferentes contas da AWS com diversos componentes.
    }
    {%
      Kubernetes clusters (EKS) administration.
      \newline
      Clusters in different AWS accounts with multiple components to be managed.
    }
  },
  {%
    \IfLanguageName{portuguese}%
    {%
      Automações para Operações em Plataforma em nível de Pipelines.
      \newline
      Estas soluções foram desenvolvidas em Shell Script e Python.
    }
    {%
      Platform automations using Piplines.
      \newline
      These solutions were coded in Shell Script and Python.
    }
  },
  {%
    \IfLanguageName{portuguese}%
    {%
      Observabilidade de sistemas e infraestrutura, como Prometheus, Grafana e Cloudwatch.
      \newline
      Métricas de sistemas, como latência, erros 5xx e 4xx, uso de memória, quantidades de requests e etc.
    }
    {%
      Systems and infrastructure observability, such as Prometheus, Grafana and Cloudwatch.
      \newline
      Metrics for systems, such as latency, 5xx and 4xx errors, memory use, amount of requests and etc.
    }
  },
  {%
    \IfLanguageName{portuguese}%
    {%
      Docker e Docker Compose para ambientes de desenvolvimento para infraestrutura.
      \newline
      Desenvolvimento de ferramenta totalmente em Shell Script para aceleramos o \href{https://github.com/raffaeldutra/cid}{desenvolvimento da Infraestrutura e disponibilizado como Open Source que pode ser encontrado em https://github.com/raffaeldutra/cid}.
    }
    {%
      Docker and Docker Compose for development and infrastructure environments.
      \newline
      Development of a tool totaly written in Shell Script to speed up our \href{https://github.com/raffaeldutra/cid}{infrastructure development and released as Open Source. The project can be found at https://github.com/raffaeldutra/cid}.
    }
 }
}

% Poatek
\def\experiencePoatek{
  {%
    \IfLanguageName{portuguese}%
    {%
      Desenvolvimento completo na AWS com Continuous Integration e Continuous Delivery/Deployment com CodeCommit, CodeBuild, CodePipeline, Cloudwatch, XRay, Lambda Functions e CloudFormation.
      \newline
      Projeto desenvolvido para um novo Banco Colombiano onde todo o projeto foi realizado com Lambdas, sendo orquestrado pelo serviço Step Functions da AWS e totalmente monitorado pela stack de serviços do Cloudwatch.
      \newline
      Toda infraestrutura necessária, como por exemplo S3, CodePipeline e etc, foi provisionada utilizando CloudFormation.
    }
    {%
      Continuous Integration and Continuous Delivery/Deployment using CodeCommit, CodeBuild, CodePipeline, Cloudwatch, XRay, Lambda Functions and CloudFormation.
      \newline
      Project developed for a new Colombian Bank Institution. All project was created using Step Functions to orchestrate Lambda Functions and being monitored using Cloudwatch services.
      \newline
      All infrastructure needed, such as S3, CodePipeline and etc, was provisioned using CloudFormation.
    }
  },
  {%
    \IfLanguageName{portuguese}%
    {%
      Python scripts com Boto3 para serviços AWS.
      \newline
      Alguns projetos necessitavam de desenvolvimentos pontuais utilizando Boto3 para consumo dos serviços de API da AWS.
    }
    {%
      Python scripts using Boto3 for AWS services.
      \newline
      Some projects had specific development for AWS API.
    }
  },
  {%
    \IfLanguageName{portuguese}%
    {%
      Reestruturação de novos ambientes para sistemas legados utilizando Vagrant, Packer, Ansible e Terraform.
      \newline
      Projeto para sistemas legados os quais necessitavam de atualização mais frequentemente no fluxo de deployment e garantia do softwares ao rodar nos diferentes ambientes, como DEV e QA.
      \newline
      Para os times de desenvolvimento foram criados ambientes em Docker, Docker Compose e Shell Scripts para automações de Infraestrutura.
    }
    {%
      Development environment automation, such as Vagrant, Packer, Ansible, and Terraform.
      \newline
      Project for legacy systems. Those environments had to be updated more frequently and with guarantee to be deployed in different environments, such as DEV and QA.
      \newline
      For the development teams, environments were created using Docker, Docker Compose and Shell Scripts for Infrastructure automation.
      \newline
    }
  }
}

% Sicredi
\def\experienceSicredi{
  {%
    \IfLanguageName{portuguese}%
    {%
      Planejamento, automação e implementação de infraestrutura em ambiente AWS.
      \newline
      Projetos internos dos mais variados utilizando como stack Packer, Terraform e Ansible.
    }
    {%
      Planning, automation and implementation of infrastructure for AWS environments.
      \newline
      Some internal projects using as stack tools like Packer, Terraform and Ansible.
    }
  },
  {%
    \IfLanguageName{portuguese}%
    {%
      Continuous Integration e Continuous Delivery/Deployment com Gitlab.
      \newline
      Utilizado para ambiente on-premise e provedor de nuvem pública.
    }
    {%
      Continuous Integration and Continuous Delivery/Deployment using Gitlab.
      \newline
      Used for on-premises and public cloud providers.
    }
  },
  {%
    \IfLanguageName{portuguese}%
    {%
      Ambientes Blue/Green para ambientes AWS.
      \newline
      Projeto realizado utilizando instâncias EC2 com controle de peso via DNS para redirecionar tráfego entre os dois ambientes para novos deployments.
    }
    {%
      Blue/Green AWS environments.
      \newline
      Project created using EC2 instances with weight controle via DNS for redirect traffic between both environments for new deployments.
    }
  },
  {%
    \IfLanguageName{portuguese}%
    {%
      Coleta de métricas com Spring Boot para Hashicorp Consul.
      \newline
      Projeto desenvolvido usando SprintBoot para coletar métricas de Hashicorp Consul e enviá-las para o Telegraf e serem visualizadas no Grafana.
    }
    {%
      Metrics for Hashicorp Consul using Spring Boot.
      \newline
      Project developed using SpringBoot to collect metrics from Consul and send it to Telegraf and visualized in Grafana.
    }
  },
  {%
    \IfLanguageName{portuguese}%
    {%
      Ansible Roles para automação de cluster multi-master para Kubernetes.
      \newline
      Projeto para automação de cluster multi-master e adição das máquinas workers on-premise com Ansible. Todas as roles eram testadas utilizando o projeto Testinfra.
    }
    {%
      Ansible Roles to automate multi-master Kubernetes cluster.
      \newline
      Project for cluster multi-master automation and workers machines add on-premise with Ansible. All Ansible roles were tested using Testinfra project.
    }
  },
  {%
    \IfLanguageName{portuguese}%
    {%
      Python scripts com Boto3 para serviços AWS.
      \newline
      Alguns projetos necessitavam de desenvolvimento diretamente com consumo de API da AWS.
    }
    {%
      Python script using Boto3 for AWS services.
      \newline
      Some projects had to be developed directly using AWS API.
    }
  }
}

% Stefanini
\def\experienceStefanini{
  {%
    \IfLanguageName{portuguese}%
    {%
      Infraestrutura como código com Terraform e Cloudformation.
    }
    {%
      Infrastructure as code using Terraform and CloudFormation.
    }
  },
  {%
    \IfLanguageName{portuguese}%
    {%
      Continuous Integration e Continuous Delivery/Deployment com Gitlab, CodeCommit, CodeBuild e CodePipeline.
    }
    {%
      Continuous Integration and Continuous Delivery/Deployment using Gitlab, CodeCommit, CodeBuild and CodePipeline.
    }
  },
  {%
    \IfLanguageName{portuguese}%
    {%
      Administração de servidores on-premises com Puppet.
    }
    {%
      On-premises servers administration with Puppet.
    }
  },
  {%
    \IfLanguageName{portuguese}%
    {%
      Empacotamento de Softwares com Packer, Ansible e Gitlab para publicação de novas AMIs na AWS para projetos rodando em modo Blue/Green na AWS.
    }
    {%
      Packer, Ansible and Gitlab to create AMIs images to be published in projects with Blue/Green environments.
    }
  }
}

% Jive
\def\experienceJive{
  {%
    \IfLanguageName{portuguese}%
    {%
      Infraestrutura como código com Terraform.
    }
    {%
      Infrastructure as code using Terraform.
    }
  },
  {%
    \IfLanguageName{portuguese}%
    {%
      Rundeck para configuração e agendamento de serviços para usuários.
    }
    {%
      Rundeck to create configuration and services to automate tasks for specific users.
    }
  },
  {%
    \IfLanguageName{portuguese}%
    {%
      Continous Integration com Bitbucket e Jenkins.
    }
    {%
      Continuous Integration with Bitbucket and Jenkins.
    }
  },
  {%
    \IfLanguageName{portuguese}%
    {%
      Logs e métricas com Datadog e Flapjack.
    }
    {%
      Logs and metrics with Datadog and Flapjack.
    }
  },
  {%
    \IfLanguageName{portuguese}%
    {%
      Documentar processos e ferramentas que eram desenvolvidas internamente para os clientes na AWS.
    }
    {%
      Documentation for internal processes and tools developd internally for AWS clients.
    }
  }
}

% AGCO
\def\experienceAgco{
  {%
    \IfLanguageName{portuguese}%
    {%
      Administração de servidores on-premises com Puppet.
    }
    {%
      On-premises servers administration with Puppet.
    }
  },
  {%
    \IfLanguageName{portuguese}%
    {%
      Logs e métricas de monitoramento Elasticsearch, Logstash, Grafana, Telegraf e InfluxDB.
    }
    {%
      Logs, metrics and monitoring with Elasticsearch, Logstash, Grafana Telegraf and InfluxDB.
    }
  },
  {%
    \IfLanguageName{portuguese}%
    {%
      Automação para ambientes de desenvolvimento com Vagrant, Packer, Linux, Preseed.
    }
    {%
      Automation for development environments with Vagrant, Packer, Linux, Preseed.
    }
  },
  {%
    \IfLanguageName{portuguese}%
    {%
      Continuous Integration e Continuous Delivery/Deployment com Bamboo.
    }
    {%
      Continuous Integration and Continuous Delivery/Deployment com Bamboo.
    }
  }
}

\section{
  \IfLanguageName{portuguese}%
  {Experiência}
  {Experience}
  }
  \vspace{3pt}
  \resumeSubHeadingListStart

    % Liber Capital
    \resumeSubheading
      {Liber Capital}{\cityPortoAlegre}
      {DevOps Engineer}%
        {%
          \IfLanguageName{portuguese}%
          {Dez 2023 \textbf{--} \cj}
          {Dec 2023 \textbf{--} \cj}
        }

        \resumeItemListStart

          \foreach \x in \experienceLiberCapital
          {
            \resumeItem{\x}
          }

        \resumeItemListEnd

    \resumeSubheading
      {4ALL}{\cityPortoAlegre}
      {DevOps Engineer}%
        {%
          \IfLanguageName{portuguese}%
          {Fev 2022 \textbf{--} Jul 2023}
          {Feb 2022 \textbf{--} Jul 2023}
        }

        \resumeItemListStart

          \foreach \x in \experienceFourAll
          {
            \resumeItem{\x}
          }

        \resumeItemListEnd

    \resumeSubheading
      {Poatek}{\cityPortoAlegre}
      {DevOps Engineer}{Jan 2021 \textbf{--} Jan 2022}
        \resumeItemListStart

          \foreach \x in \experiencePoatek
          {
            \resumeItem{\x}
          }

        \resumeItemListEnd

    \resumeSubheading
      {Sicredi}{\cityPortoAlegre}
      {DevOps Engineer}%
        {%
          \IfLanguageName{portuguese}%
          {Ago 2019 \textbf{--} Jan 2021}
          {Aug 2019 \textbf{--} Jan 2021}
        }
        \resumeItemListStart

          \foreach \x in \experienceSicredi
          {
            \resumeItem{\x}
          }

        \resumeItemListEnd

    \resumeSubheading
      {Stefanini}{\cityPortoAlegre}
      {DevOps Engineer}%
        {%
          \IfLanguageName{portuguese}%
          {Ago 2018 \textbf{--} Mai 2019}
          {Aug 2018 \textbf{--} May 2019}
        }
        \resumeItemListStart

          \foreach \x in \experienceStefanini
          {
            \resumeItem{\x}
          }

        \resumeItemListEnd

    \resumeSubheading
      {Jive Softwares}{\cityPortland}
      {DevOps Engineer}%
        {%
          \IfLanguageName{portuguese}%
          {Jan 2018 \textbf{--} Jan 2018}
          {Jan 2018 \textbf{--} Jul 2018}
        }
        \resumeItemListStart

          \foreach \x in \experienceJive
          {
            \resumeItem{\x}
          }

        \resumeItemListEnd

    \resumeSubheading
      {AGCO}{\cityCanoas}
      {DevOps Engineer}%
        {%
          \IfLanguageName{portuguese}%
          {Out 2016 \textbf{--} Jan 2018}
          {Oct 2016 \textbf{--} Jan 2018}
        }
        \resumeItemListStart

          \foreach \x in \experienceAgco
          {
            \resumeItem{\x}
          }

        \resumeItemListEnd

  \resumeSubHeadingListEnd
